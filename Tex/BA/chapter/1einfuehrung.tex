
\label{cha:einfuehrung}
Diese Arbeit beschäftigt sich mit dem Design und der Implementierung eines Plugins für die Lernplattform Moodle, welches es ermöglichen soll, den Studierenden die Lerninhalte mittels Hyperaudio-Dokumenten bereitzustellen. Ziel dieses Plugins ist die Erweiterung der Lernmöglichkeiten an der FernUniversität in Hagen, um die Studierenden beim Erreichen ihrer Lernziele besser zu unterstützen. 


%%%%%%%%%%
\section{Motivation}
\label{sec:motivation}
Die Motivation zur Behandlung dieses Themas besteht darin, dass 
%ca. 80\% der Studierenden in Teilzeit studieren und 
80\% der Studierenden gleichfalls neben dem Studium arbeiten \citep{fernuniversitaet2018stat}. Unter diesen Umständen beschäftigen sich viele Studierende erst kurz vor der Prüfung - dafür aber entsprechend intensiv - mit den Lerninhalten. An der Fakultät Mathematik/Informatik und auch an der Fakultät für Wirtschaftswissenschaften bestehen diese Lerninhalte zu einem guten Teil aus textlastigen Kurseinheiten, die Abbildungen und Formeln enthalten.


Diese Aussage kann anhand der Pflichtmodule des Bachelorstudiengangs Wirtschaftsinformatik bestätigt werden.
Die Pflichtmodule an der Faktultät Mathematik/Informatik weisen einen Textanteil von XXX auf. An der der Fakultät für Wirtschaftswissenschaften liegt der Anteil in den Pflichtmodulen nochmals höher bei XXX. Die Analyse der Kurseinheiten wurde mittels der Textanalyse des Programms \textit{PDF-Analyzer 5.0}\footnote{http://www.is-soft.de} vorgenommen.
\todo[inline]{Abschnitt nach Abschluss der Analyse der Kurseinheiten aktualisieren}

Eher selten werden auch Videos angeboten, in welchen bestimmte Lerninhalte aus den Kurseinheiten nochmals rekapituliert werden. Hier sind als Beispiel die Videos von Univ.-Prof. Dr. Ulrike Baumöl zum Kurs \glqq Informationsmanagement\grqq{} zu nennen.

Die Idee besteht nun darin, den Lernenden erstens eine alternative Repräsentation (Modalität) der Lerninhalte anzubieten und ihnen zweitens das Lernen während ungenutzter Alltagssituationen zu ermöglichen (z.B. lange Autofahrten, Pendeln in Bus und Bahn, beim Joggen, etc.). Auf diese Weise könnten die Lernenden die Inhalte häufiger rezipieren und einüben. Ergänzt um gute E-Assessments (Selbsttests) hätten Sie in Summe eine Chance, sich frühzeitig und kontinuierlich auf die Prüfung vorzubereiten und vielleicht bessere Lernerfolge zu erzielen. 


%%%%%%%%%
\section{Problemstellung}
Diese Arbeit wird sich in diesem Zusammenhang vor allem mit dem Problem des hohen Textanteils vieler Kurse und dem damit verbundenen Lernverhalten beschäftigen. Doch zunächst soll die Ist-Situation für die Studierenden an der FernUniversität in Hagen beschrieben werden.

%%%%%%%%%
\subsection{Ist-Situation}
Jeder Studierende hat Zugriff auf den \textit{Virtuellen Studienplatz}, häufig auch \textit{Virtuelle Universität} (VU) oder \textit{Lernraum Virtuelle Universität} (LVU) genannt. Hierbei handelt es sich um ein eigenentwickeltes Webportal der FernUniversität in Hagen. Der \textit{Virtuelle Studienplatz} stellt unter anderem das Kurs- und Studiumsportal der FernUniversität dar. Hierüber können die Studierenden Kurse belegen, die Rückmeldung für das nächste Semester vornehmen oder ihre persönlichen Daten einsehen und bearbeiten. Auch eine Übersicht über das Veranstaltungsangebot wird dem Studierenden geboten. Zusätzlich bietet der \textit{Virtuelle Studienplatz} eine Übersicht über alle belegten Kurse des Studierenden \citep{fernuniversitaet2018vu}. Mittels dieser Übersicht kann der Studierende direkt auf das jeweilige persönliche Kursportal seiner belegten Kurse gelangen. Neben allgemeinen Inforamtionen zu dem Kurs bietet das Kursportal unter anderem Zugriff auf die Online-Studieninhalte (z.B. Kurseinheiten\footnote{Kurseinheiten werden an der FernUniversität in Hagen auch als Studienbriefe bezeichnet.}, Einsendeaufgaben, Musterlösungen) und Verweise zu anderen Diensten (Moodle, Online-Übungssystem, Kommunikationsangebote Adobe Connect Videokonferenzen), die in diesem Kurs zur Verfügung stehen \citep{fernuniversitaet2018kurs}.

Neben der Möglichkeit, die Kurseinheiten als PDF über den \textit{Virtuellen Studienplatz} herunterzuladen, werden diese im Regelfall für die belegten Kurse automatisch in gedruckter Form an die Studierenden versendet. Die Kurseinheiten dienen als zentrales Lernmaterial für die Studierenden.

Jeder Kurs hat die Möglichkeit, zusätzliches Lernmaterial über Moodle, die zentrale Lernplattform der FernUniversität in Hagen, zur Verfügung zu stellen. Moodle bietet den Kursen \glqq als sogenanntes Learningmanagementsystem (LMS) vielfältige Möglichkeiten zur Gestaltung der mediengestützten Lehre an\grqq{} \citep{fernuniversitaet2018moodle}. Besonders hervorzuheben ist hierbei der mögliche Einsatz von Lehrvideos, Foren und Tests. Die Platttform Moodle und ihre Erweiterungsmöglichkeiten werden in Abschnitt \ref{sec:moodle} vorgestellt. Im Bezug auf die Pflichtmodule des Bachelorstudiengangs Wirtschaftsinformatik an der FernUniversität in Hagen wird Moodle in 21 von 30 Kursen beziehungsweise 10 von 15 Modulen eingesetzt (vgl. Anhang \ref{fig:VerwendungMoodle}). \textbf{Auffällig ist bei den Pflichtmodulen des Bachelorstudiengangs Wirtschaftsinformatik, dass Moodle an der Fakultät für Wirtschaftswissenschaft für jeden Kurs angeboten wird, während an der Fakultät für Mathematik und Informatik kein einziger Kurs auf Moodle zurückgreift.}
\todo[inline]{An der Fakultät gibt es mindestens vier Moodle-Kurse.}

Durch den Einsatz von Adobe Connect besteht an der FernUniversität in Hagen auch die Möglichkeit sogenannter \textit{Virtual Classrooms}. Dabei handelt es sich um eine Video- und Tonübertragung mit Textchat und Freigabemöglichkeiten für Präsentationen und Bildschirminhalte. \glqq Ein \emph{Virtual Classroom} [Hervorhebung v. Verf.] eignet sich insbesondere für Veranstaltungen, in denen die synchrone Kommunikation ein hohes Gewicht erhält: Seminare, Tutorien, Sprechstunden, Arbeitsgruppen u.Ä.\grqq{} \citep{fernuniversitaet2018kommunikationstools}.

Darüber hinaus werden den Studierenden mit den \textit{Diskussionsforen} (Newsportal) und dem \textit{Conference Center} als Chat-System zwei weitere Systeme zur Kommunikation geboten \citep{fernuniversitaet2018kommunikationstools}.


%%%%%%%%%
\subsection{Probleme}
Die im vorangegangen Abschnitt beschriebenen Lernangebote weisen jedoch Defizite im Bezug auf die Vermittlung der Lerninhalte auf. Der \textit{Virtuelle Studienplatz} dient aktuell ausschließlich als Portal, um die Studierenden zu den von ihnen benötigen Informationen zu leiten und unterstützt somit nur indirekt die Vermittlung von Lerninhalten.

Die Kurseinheiten als zentrales Lernmaterial bestehen, wie anhand der Zahlen aus Abschnitt \ref{sec:motivation} erkenntlich, zum Großteil aus Text. Dies hat zur Folge, dass sich die Studierenden während der Auseinandersetzung mit den Lerninhalten nicht mit anderen Dingen beschäftigen können, welche die Aufmerksamkeit ihrer Augen und Hände benötigen. Zusätzlich besteht oft das Problem, dass Abbildungen, Formeln und Tabellen, auf die im Text verwiesen wird, nicht direkt auf der Seite ersichtlich sind, auf der diese im Text erwähnt werden. Hierdurch ist oftmals Blättern bzw. Scrollen nötig, je nachdem ob die Kurseinheit in Papierform oder digital bearbeitet wird. Dies erschwert zusätzlich das Verinnerlichen des in der Kurseinheit zu vermittelnden Inhalts. Im Vergleich zum Frontalunterricht bringt die Vermittlung der Lerninhalte in Form von Kurseinheiten den Nachteil mit sich, dass bei Verständnisproblemen keine direkten Fragen gestellt werden können.

Dieses Problem tritt bei Lehrvideos ebenso auf. Hier ist im besten Fall ein asynchroner Austausch mittels einer Kommentarfunktion möglich. Ähnlich wie Kurseinheiten verlangen auch Lehrvideos die durchgehende visuelle Aufmerksamkeit des Studierenden. Nur durch ununterbrochenes Betrachten eines Videos kann ein Studierender sicherstellen, dass er alle dargestellten Inhalte wahrnimmt.

Im Gegensatz zu Lehrvideos, sind die Foren und der Chat ohnehin nur als zusätzliche Kommunikationswege für die Studierenden implementiert. Es besteht das grundsätzliche Problem, dass diese Funktionalitäten nicht direkt an die Lerninhalte gekoppelt sind und deswegen separat aufgerufen werden müssen, falls beim Lernen Fragen auftreten sollten. 

Die in Moodle verfügbaren Tests dienen hingegen in ihrer Form nur zur reinen Selbstkontrolle. Dadurch, dass diese Tests nicht direkt während der Erarbeitung der Lerninhalte durch Kurseinheit oder Lehrvideos erfolgen kann, wird nicht unmittelbar überprüft, ob die Inhalte korrekt verstanden wurden.

\textit{Virtual Classrooms} zählen wiederum zu den Lerninhalt vermittelnden Angeboten. Bedingt durch die Tatsache, dass der Unterricht in \textit{Virtual Classrooms} in Echtzeit abgehalten wird, entsteht der Nachteil, dass der Studierende nur zu einem festgelegten Zeitpunkt die Lehrveranstaltung wahrnehmen kann. Außerdem verlangen \textit{Virtual Classrooms}, genau wie Lehrvideos, die ständige Aufmerksamkeit des Studierenden. Im Gegensatz zu Lehrvideos schaffen \textit{Virtual Classrooms} jedoch die Möglichkeit der synchronen Kommunikation.

Zusammenfassend handelt es sich bei den Kurseinheiten, Lehrvideos und \textit{Virtual Classrooms} um  die Lerninhalt vermittelnden Angebote. Kurseinheiten und Lehrvideos stellen hierbei asynchrone Lehrmethoden dar, während es sich bei den \textit{Virtual Classrooms} um eine synchrone Lehrmethode handelt. Die asynchronen Angebote bringen im Gegensatz zum synchronen Angebot den Vorteil mit sich, dass diese zu jeder beliebigen Zeit wahrgenommen werden können. Dennoch haben diese drei Lehrangebote gemeinsam, dass die Studierenden ihnen zumindest die volle visuelle Aufmerksamkeit beim Lernen schenken müssen. 

Somit ist mit dem aktuellen Lehrangebot beispielsweise auch kein Lernen während der sportlichen Betätigung möglich. Dabei führt leichte körperliche Betätigung während des Lernens nach einer Studie von \cite{schmidt2013physical} sogar zu einem besseren Lernergebnis. Stattdessen ist der Studierende weiterhin daran gebunden im Sitzen oder gar vor dem Bildschirm zu lernen. Indes haben mehrere Studien gezeigt, dass langes Sitzen negative Auswirkungen auf die gesundheitliche Verfassung mit sich bringt \citep{tremblay2011systematic}.

Aufgrund der Tatsache, dass 80\% der Studierenden an der FernUniversität in Hagen neben dem Studium ebenfalls einer Arbeit nachgehen \citep{fernuniversitaet2018stat}, muss auch die dem Studierenden zur Verfügung stehenden Zeit berücksichtigt werden. Zur bezahlten Arbeit kommt immer auch unbezahlte Arbeit hinzu. Diese betrug in den Jahren 2012/2013 im Durchschnitt ca. 24,5 Stunden in der Woche für Personen ab 18 Jahren. Als unbezahlte Arbeit gelten Haushaltstätigkeiten, wie Kochen, Putzen, Gartenpflege und Einkaufen, aber auch ehrenamtliche Tätigkeiten sowie Wegzeiten \citep{destatis2015zeit}. Diese unbezahlte Arbeit kann aktuell zum Großteil nicht zum Lernen verwendet werden, da durch die heutigen Lernangebote stets die volle Aufmerksamkeit des Studierenden erforderlich ist. Beispielsweise ist es unmöglich, während des Fensterputzes eine Kurseinheit in ihrer aktuellen Repräsentation zu lesen, da die Putztätigkeit bereits die volle visuelle Aufmerksamkeit sowie den Einsatz der Hände erfordert. Diese Zeit könnte jedoch zum Lernen genutzt werden, sofern die Lerninhalte in einer anderen Form präsentiert würden, die sich auf andere Sinne beschränkt. Da der Hörsinn in vielen Alltagstätigkeiten, wie dem Putzen oder Einkaufen, nicht vorrangig benötigt wird, bietet sich dazu eine Repräsentation in auditiver Form an.

%- Welche Modalität der Lernmaterialien ist für meine aktuelle körperliche/psychische Verfassung in meiner aktuellen Umgebung am besten geeignet?
%
%- Wann habe ich wie viel Zeit um mir die Materialien anzusehen/anzuhören?
%
%- In der Argumentation geht es etwas durcheinander zu. Sie müssen klarer unterscheiden, z.B. zw. synchronem und asynchronem Lernen.
%
%\todo[inline]{Arbeiten Sie noch besser heraus, warum und wann das Lernen mit visuellen Medien für Lernende nicht möglich ist. Es geht in der Argumentation nicht darum, welche Modalität die beste ist, sondern um die Frage wie man die Lernmaterialien so aufbereiten kann, dass Lernenden das Lernen besser in ihren Alltag integrieren können. Der Aspekt der unbezahlten Arbeit ist ein gutes Argument! Welche Modalität der Lernmaterialien ist für meine aktuelle körperliche/psychische Verfassung in meiner aktuellen Umgebung am besten geeignet? Man kann dies auch anhand der für das Lernen verfügbaren Zeit diskutieren: Wann habe ich wie viel Zeit um mir die Materialien anzusehen/anzuhören? Welche Probleme ergeben sich daraus, wenn ich versuche, alle Materialien als Audio aufzubereiten? Darüber hinaus können Lerntypen eine Rolle spielen. Ob jedoch Lerntypen wirklich existieren, wird von Forschern vielfach angezweifelt.  
%In der Argumentation geht es etwas durcheinander zu. Sie müssen klarer unterscheiden, z.B. zw. synchronem und asynchronem Lernen. Es scheint klar, dass die mit der Themenstellung verbundene Ermöglichung von Flexibilität beim Lernen nicht mit synchronen Lernszenarien vereinbar ist. }


%%%%%%%%%
\section{Zielsetzung und Forschungsfragen}
\label{sec:zielsetzung}
Hauptziel des Vorhabens ist die Gestaltung einer auditive Repräsentation der Lerninhalte und der Entwicklung eines Plugins innerhalb von Moodle zur Wiedergabe dieser alternativen Repräsentation. Die Wiedergabe soll in die Moodle-Plattform integriert werden, da diese mit ihrem Plugin-System und dem hohen Verbreitungsgrad an der FernUniversität in Hagen gute Voraussetzungen für die Bereitstellung neuer Lehrmethoden bietet.
Im Laufe dieser Arbeit soll die Frage beantwortet werden, wie eine Hyperaudio-Lernumgebung in Moodle zu gestalten ist. Dabei ergeben sich folgende untergeordnete Forschungsfragen:

\begin{itemize}
\item Wie kann den Studierenden mithilfe einer Hyperaudio-Lernumgebung ermöglicht werden mehr Zeit zum Lernen nutzen zu können?
\item Wie lassen sich Inhalte in der Hyperaudio-Lernumgebung darstellen, die nicht in auditiver Form abgebildet werden können?
\item \textbf{Wie lassen sich auditive Inhalte verständlich gestalten?}
\item Wie können nicht-auditive Inhalte mit den auditiven Inhalten verknüpft werden?
\item Wie lassen sich alle Interaktionen, die eine textuelle Darstellung der Lerninhalte bietet, in der Hyperaudio-Lernumgebung umsetzen?
\item Wie kann der Austausch zwischen Studierenden und Lehrenden umgesetzt werden?
\end{itemize}

Ziel dieser Arbeit soll es also sein, den Studierenden eine auditive Repräsentation der Lerninhalte anzubieten, welche es ihnen ermöglichen soll mehr Zeit zum Lernen nutzen zu können und dabei die Effizienz des Lernens zu erhöhen. 

Es muss eine Lernumgebung gestaltet werden, welche es ermöglicht, die hauptsächlich auditiven Lerninhalte bereitzustellen. Daneben müssen auch Lerninhalte, welche nicht in auditiver Form abgebildet werden können, berücksichtigt werden. Akustische Signale können unterstützend eingesetzt werden, um Studierende auf Zusatzinhalte aufmerksam zu machen, die ihre tiefergehende Aufmerksamkeit erfordern. Typische Nutzerinteraktionen mit textuellen Lernmedien, wie beispielsweise persönliche Notizen und Lesezeichen oder das Durchsuchen des Inhaltsverzeichnisses, sollen hierbei erhalten bleiben.

Zusätzlich zur Bereitstellung der Lerninhalte in alternativer Form soll die Hyperaudio-Lernumgebung auch den Kommunikationsaustausch zwischen Lehrenden und Studierenden ermöglichen und fördern. Hiermit soll dem allgemeinen Problem des mangelnden sozialen Kontakts beim Fernstudium begegnet werden \citep{kerres2002didaktische}. Um den Studierenden beim Lernen möglichst große Flexibilität zu bieten, ist eine mobile Verfügbarkeit der neugestalteten Lerninhalte erstrebenswert.

\todo[inline]{Daneben müsst man auch fragen, wie sich auditive Inhalte darstellen lassen. Denken Sie an meine Mail mit den Ideen, die mir während der Aufnahme der Kurseinheiten gekommen sind.}



%Hier mal ein paar Phrasen:
%- Hauptziel des Vorhabens ist die  Gestaltung, ... und Entwicklung einer ...
%- Didaktisches Ziel ist es daher, den Lernenden 
%- Aus technischer Sicht besteht das Ziel darin ...