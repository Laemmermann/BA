%%%%%%%%%%
\section{Architektur des Moodle-Plugins}
Bei der Implementierung des Moodle-Plugins ist die durch Moodle vorgegebene Architektur von Plugins zu beachten. Diese besteht stets aus vorgegebenen Dateien und Ordnen, die Anzahl ist jeweils von der Art des zu entwickelnden Plugins abhängig. Des Weiteren bestimmt die Art des Plugins auch den zu wählenden Speicherort.

Bei Activity Plugins, wie dem Plugin für Hyperaudio-Dokumente, ist als Speicherort der Ordner \textbf{/mod} vorgeben. In diesem Ordner muss ein Unterordner mit dem Namen des Plugins angelegt werden. Darin werden wiederum die verpflichtenden Dateien und Ordner abgespeichert.
\todo[inline]{verpflichtenden?}

Im Ordner \textbf{backup} werden die Dateien abgelegt, welche Anwendung finden, wenn ein Backup oder eine Wiederherstellung eines Kurses vorgenommen wird.

Der Ordner \textbf{db} beherbergt die Dateien \textbf{access.php}, \textbf{install.xml} und \textbf{upgrade.php}. Die Datei \textbf{access.php} dient zur Steuerung der Berechtigungen innerhalb des Moodle-Plugins, wobei den verschiedenen Moodle-Rollen verschiedene Rechte für die einzelnen Funktionen zugewiesen werden können. Bei der Installation des Plugins wird die \textbf{install.xml} zur Erstellung der Datenbanktabellen für das Plugin verwendet. Es ist mindestens eine Tabelle mit dem Namen des Plugins anzulegen. Sollten die Datenbanktabellen nach Veröffentlichung des Plugins um Spalten erweitert werden, so kommt die Datei \textbf{upgrade.php} zum Einsatz. Hierin werden die notwendigen Schritte für einen Versionsabgleich definiert.

Die Sprachdateien werden im Ordner \textbf{lang} und dem jeweiligen Unterordner für die jeweilige Sprache abgelegt. Pro Sprache wird eine \textit{PHP}-Datei mit dem Namen des Plugins gespeichert, hierin befindet sich dann die jeweilige Lokalisierung.
\todo[inline]{gna...}

Das Icon, welches für das Plugin  verwendet werden soll, muss im Ordner \textbf{pix} mit dem Dateinamen \textbf{icon.gif} abgelegt werden und sollte eine Auflösung von 16x16 Pixel besitzen.

lib.php

mod\_form.php

index.php

view.php

version.php

\todo[inline]{Zitat von Moodle Seite einfügen}

\todo[inline]{Ordnerstruktur}

%%%%%%%%%%
\section{Iterative Entwicklung}
\dots

\subsection{Speichern und Abspielen einer Audio-Datei}
\dots

\subsection{Speichern und Anzeige von Zusatzinhalten}
\dots

\subsection{Einbindung der Konfigurationsdatei}
\dots

\subsection{Speichern und Anzeige von Kommentaren}
\dots

\subsection{Antworten auf Kommentare}
\dots

\subsection{Galerie der Zusatzinhalte}
\dots

\subsection{Visualisierung der Annotationen in der Timeline}
\dots

%%%%%%%%%%
\section{Zusammenfassung}
\dots