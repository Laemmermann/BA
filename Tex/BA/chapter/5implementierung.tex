%%%%%%%%%%
\section{Architektur des Moodle-Plugins}
Bei der Implementierung des Moodle-Plugins ist die durch Moodle vorgegebene Architektur von Plugins zu beachten. Diese besteht stets aus vorgegebenen Dateien und Ordnen, die Anzahl ist jeweils von der Art des zu entwickelnden Plugins abhängig. Des Weiteren bestimmt die Art des Plugins auch den zu wählenden Speicherort.

Bei Activity Plugins, wie dem Plugin für Hyperaudio-Dokumente, ist als Speicherort der Ordner \textbf{/mod} vorgeben. In diesem Ordner muss ein Unterordner mit dem Namen des Plugins angelegt werden. Darin werden wiederum die \textbf{vorgegebenen} Dateien und Ordner abgespeichert.
\todo[inline]{verpflichtenden?}

Im Ordner \textbf{backup} werden die Dateien abgelegt, welche Anwendung finden, wenn ein Backup oder eine Wiederherstellung eines Kurses vorgenommen wird.

Der Ordner \textbf{/mod/db} beherbergt die Dateien \textbf{access.php}, \textbf{install.xml} und \textbf{upgrade.php}. Die Datei \textbf{access.php} dient zur Steuerung der Berechtigungen innerhalb des Moodle-Plugins, wobei den verschiedenen Moodle-Rollen verschiedene Rechte für die einzelnen Funktionen zugewiesen werden können. Bei der Installation des Plugins wird die \textbf{install.xml} zur Erstellung der Datenbanktabellen für das Plugin verwendet. Es ist mindestens eine Tabelle mit dem Namen des Plugins anzulegen. Sollten die Datenbanktabellen nach Veröffentlichung des Plugins um Spalten erweitert werden, so kommt die Datei \textbf{upgrade.php} zum Einsatz. Hierin werden die notwendigen Schritte für einen Versionsabgleich definiert.

\textbf{Im Ordner \textbf{/mod/lang} wird die Sprachlokalisierung abgespeichert. Für jede Sprache wird innerhalb des \textbf{lang} Ordners ein eigener Unterordner angelegt. In diesem wird dann jeweils eine \textit{PHP}-Datei mit dem Namen des Plugins gespeichert, in welcher die jeweilige Lokalisierung vorgenommen wird.}

Die Sprachdateien werden im Ordner \textbf{lang} und dem jeweiligen Unterordner für die jeweilige Sprache abgelegt. Pro Sprache wird eine \textit{PHP}-Datei mit dem Namen des Plugins gespeichert, hierin befindet sich dann die jeweilige Lokalisierung.
\todo[inline]{gna...}

Das Icon, welches für das Plugin  verwendet werden soll, muss im Ordner \textbf{/mod/pix} mit dem Dateinamen \textbf{icon.gif} abgelegt werden und sollte eine Auflösung von 16x16 Pixel besitzen.

Im Ordner \textbf{/mod} liegen dann die Dateien \textit{lib.php}, \textit{mod\_form.php}, \textit{index.php}, \textit{view.php} und \textit{version.php}. Die \textit{lib.php} dient dazu Standardfunktionen von Moodle zu überschreiben, hier sind die \textit{add\_instance}, \textit{update\_instance} und \textit{delete\_instance} Funktionen als essentielle Funktionen zu nennen. Mit diesen Funktionen wird das Anlegen, Aktualisieren und Löschen von Inhalten des Plugins ermöglicht. Zum Anlegen und Aktualisieren wird in der \textit{mod\_form.php} die dazugehörige Maske festgelegt.



\textbf{\textit{index.php}}

Die \textit{view.php} ist die erste Datei die beim Öffnen der Aktivität geöffnet wird und dient dementsprechend vornehmlich der Anzeige der Inhalte. In der \textit{version.php} wird die Version des Plugins gepflegt, diese Datei ist wichtig um den Upgrade Prozess anzustoßen.

\todo[inline]{Zitat von Moodle Seite einfügen}

\todo[inline]{Ordnerstruktur}

%%%%%%%%%%
\section{Iterative Entwicklung}
Die Entwicklung des Plugins wird in iterativer Form durchgeführt. In jeder Iteration soll das Plugin nur um einige wenige Funktionalitäten erweitert werden.  

\subsection{Speichern und Abspielen einer Audio-Datei}
In der ersten Iteration wurde die grundlegende Struktur des Plugins erstellt
\subsection{Speichern und Anzeige von Zusatzinhalten}
\dots

\subsection{Einbindung der Konfigurationsdatei}
\dots

\subsection{Speichern und Anzeige von Kommentaren}
\dots

\subsection{Antworten auf Kommentare}
\dots

\subsection{Galerie der Zusatzinhalte}
\dots

\subsection{Visualisierung der Annotationen in der Timeline}
\dots

%%%%%%%%%%
\section{Zusammenfassung}
\dots