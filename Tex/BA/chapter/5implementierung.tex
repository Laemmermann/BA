%%%%%%%%%%
\section{Architektur des Moodle-Plugins}
Bei der Implementierung des Moodle-Plugins ist die generelle durch Moodle vorgegebene Architektur zu beachten. Diese besteht stets aus vorgegebenen Dateien und Ordnen, die Anzahl ist jeweils von der Art des zu entwickelnden Plugins abhängig.


%%%%%%%%%%
\section{Iterative Entwicklung}
\dots

\subsection{Speichern und Abspielen einer Audio-Datei}
\dots

\subsection{Speichern und Anzeige von Zusatzinhalten}
\dots

\subsection{Einbindung der Konfigurationsdatei}
\dots

\subsection{Speichern und Anzeige von Kommentaren}
\dots

\subsection{Antworten auf Kommentare}
\dots

\subsection{Galerie der Zusatzinhalte}
\dots

\subsection{Visualisierung der Annotationen in der Timeline}
\dots

%%%%%%%%%%
\section{Zusammenfassung}
\dots