Zusammenfassend wurden in dieser Arbeit, basierend auf einer mittels Personas und User Stories entwickelten Zielgruppe, Anforderungen an ein Plugin gestellt, welches eine auditive Repräsentation der bisherigen Kurseinheiten an der FernUniversität in Hagen darstellen soll. Aufbauend auf diesen Anforderungen wurde der aktuelle Stand der Technik betrachtet, um sowohl Ideen zu sammeln als auch eine technische Grundlage für die Umsetzung des Hyperaudio-Plugins zu finden. Im nächsten Schritt wurde die Konzeption des Plugins vorgenommen, dies bestand unter anderem daraus die verschiedenen Komponenten und deren Zusammenhänge zu definieren. Zusätzlich wurde eine Konfigurationsdatei für das Schnittstellenformat definiert als auch der Datenbankentwurf vorgenommen. Am Ende der Konzeption wurde das Design der Benutzeroberfläche erarbeitet. Basierend auf diesen konzeptionellen Entscheidungen wurde anschließend die Implementierung durchgeführt. Zum Zwecke der anschließenden Evaluierung wurden erneut die User Stories und die daraus entstanden Anforderungen in Verbindung mit der durchgeführten Implementierung betrachtet. In diesem Zuge wurden einige Schwächen, wie beispielsweise die fehlende Validierung der Konfiguration, aber auch potentielle Weiterentwicklungsmöglichkeiten aufgezeigt. 
\todo[inline]{Wurden Forschungsfragen beantwortet?}

Nachdem nun die Evaluierung erfolgt ist, kann ein Ausblick für mögliche Weiterentwicklungen gegeben werden. An dieser Stelle wären als erstes die nicht umgesetzten Anforderungen zu nennen. Um den Studierenden beim Anhören der Audio Cues zu unterstützen könnten auch die Audio Cues und die Art des Vorlesens erweitert werden. So könnte man beispielsweise für verschiedene Arten von Zusatzinhalten unterschiedliche Audio Cues abspielen, oder verschiedene Sprecher oder Audiokanäle verwenden um den Studierenden beispielsweise dabei zu unterstützen Aufzählungen besser nachvollziehen zu können. Mit solchen Ideen haben sich bereits \cite{donker2007gestaltung} auseinander gesetzt. Ein Ansatz hierfür könnte \textit{Resonance Audio}\footnote{https://developers.google.com/resonance-audio/} von Google sein.
Eine weitere Überlegung wäre es das Hyperaudio-Plugin so zu erweitern, dass es auch das Streaming auf Fernsehgeräte ermöglicht, zum Beispiel durch die Unterstützung von Google ChromeCast oder Apple TV. Hierdurch würde man die Nutzung einer großen Bildschirmfläche abseits von Laptops und Desktops ermöglichen.
Aber auch im Bereich der kleinen Displays, also der mobilen Endgeräte, besteht noch Potential zur Weiterentwicklung. Die aktuelle Implementierung beschränkt sich aktuell nur darauf, dass die Inhalte auf mobilen Endgeräte korrekt dargestellt werden. Wie bereits in der Evaluierung erwähnt, würde es sich anbieten ein speziell an die mobilen Anforderungen ausgerichtetes Design zu entwickeln. Es wäre eine App vorstellbar, welche neben einem besseren Design auch offline Funktionalitäten bietet. Damit würden die Freiheiten des Studierenden nochmals erhöht, da er nicht mehr abhängig von einer Internetverbindung ist.
Neben diesen Punkten die sich vor allem auf die Rolle der Nutzenden bezieht, wären auch einige Weiterentwicklungen für die Administrierenden denkbar. Neben der bereits in der Evaluation besprochenen Möglichkeit Hyperaudio-Dokumente komplett in Moodle erstellen zu können und somit auf die Konfigurationsdatei verzichten zu können, wäre eine Möglichkeit zur automatischen Konvertierung der klassischen Kurseinheiten in Hyperaudio-Dokumente reizvoll. Dies würde die Hürde zur Einführung von Hyperaudio-Dokumenten durch die Lehrenden enorm senken, da dies mit einem wesentlich geringeren Aufwand verbunden wäre.

\todo[inline]{Lehrend möchten außerdem den Inhalt ihrer bestehenden Kurse vollständig als Audio/Hyperaudio abbilden können. Formeln, Tabellen, Bilder,...Sie möchten bei der Konvertierung/Produktion wenig Arbeit haben. Sie möchten Inhalte nachträglich editieren (auch Audio?!).}

Ausblick:\\
Anforderungen die nicht umgesetzt wurden\\


nachträgliche Änderungen\\
Versionierung\\
\textbf{Bewertungssystem für Kommentare und Antworten, wie es zum Beispiel bei \textit{Stack Overflow}\\
Dateien bei Notizen mitspeichern}

Löschen von Kommentaren durch Administratoren\\
Videos als Zusatzinhalt
Echtzeitaustausch
Unterschiedliche Anzeige Lehrende Studierende
ScrollToTop (evtl. wenn AudioCue kommt)
Validierung (was passiert wenn Student austritt, falsches JSON File)
Export mit Kommentaren

Unterstützung für ChromeCast/Apple TV\\
https://developers.google.com/resonance-audio/\\
verschiedene Audio-Cues, über Config-JSON konfigurierbar YAY\\
mobile Endgeräte\\
Konvertierung von Kurseinheit zu Hyperaudio\\