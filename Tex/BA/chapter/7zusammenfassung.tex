Zum Abschluss dieser Arbeit sollen die Ergebnisse nochmals zusammengefasst und der Bezug zu den anfangs gestellten Forschungsfragen hergestellt werden. Abschließend soll ein Ausblick für die Weiterentwicklung des Hyperaudio"=Plugins gegeben werden.

\section{Zusammenfassung}
Beginnend mit der Formulierung der Grundgedanken für die Repräsentation der Kurseinheiten im Hyperaudio-Format, wurde das Hyperaduio-Plugin nach einer Analyse der Anforderungen und des aktuellen Stands der Technik konzeptioniert und implementiert. Dabei stand die Beantwortung der Forschungsfragen und das Erreichen der daraus abgeleiteten Ziele im Mittelpunkt.

\begin{itemize}
\item Wie kann den Studierenden mithilfe einer Hyperaudio-Lernumgebung ermöglicht werden mehr Zeit zum Lernen nutzen zu können?
\end{itemize}

Es wurde eine zusätzliche Lernumgebung für Studierende geschaffen. Durch die Umsetzung als Moodle-Plugin ist diese nahtlos in die bestehende Infrastruktur der FernUniversität in Hagen integrierbar. Da diese Hyperaudio-Lernumgebung primär auditive Inhalte vermittelt, ist das Lernen in vielen Alltagssituationen möglich. Im Vergleich zur textuellen Repräsentation von Kurseinheiten ist nur selten die visuelle Aufmerksamkeit der Studierenden gefordert. Auf diese Notwendigkeit wird mithilfe akustischer Signale hingewiesen.

\begin{itemize}
\item Wie lassen sich auditive Inhalte verständlich gestalten?
\end{itemize}

In den Grundlagen wurden Wege aufgezeigt, wie die Vertonung von Kurseinheiten durchgeführt werden kann, sodass der Studierende den Lerninhalten erfolgreich folgen kann. Dies kann unter anderem durch den Einsatz der Prinzipien für Multimedia von \cite{mayer2009multimedia} oder unter Berücksichtigung der Erkenntnisse von \cite{donker2007gestaltung} erfolgen.

\begin{itemize}
\item Wie lassen sich Inhalte in der Hyperaudio-Lernumgebung darstellen, die nicht in auditiver Form abgebildet werden können?
\end{itemize}

Nicht auditiv repräsentierte Inhalte der Kurseinheiten, wie beispielsweise Tabellen oder Grafiken, können in Form von Bildern in die Hyperaudio-Repräsentation eingebunden werden.

\begin{itemize}
\item Wie können nicht-auditive Inhalte mit den auditiven Inhalten verknüpft werden?
\end{itemize}

Das Hyperaudio"=Plugin bietet die Möglichkeit, Bilder zu definierten Zeitpunkten darzustellen. Die Zuordnung von Bildern zu Zeitpunkten erfolgt anhand einer Konfigurationsdatei im JSON-Format. Auf die eingebundenen Zusatzinhalte wird nach dem Vorschlag von \cite{donker2007gestaltung} durch Audio Cues hingewiesen.

\begin{itemize}
\item Wie lassen sich alle Interaktionen, die eine textuelle Darstellung der Lerninhalte bietet, in der Hyperaudio-Lernumgebung umsetzen?
\end{itemize}

In der Hyperaudio-Lernumgebung wurden Möglichkeiten geschaffen, um Interaktionen mit textuellen Kursrepräsentationen digital nachzubilden. So kann das klassische Lesezeichen in der Kurseinheit durch ein Lesezeichen in der Timeline des Hyperaudio-Dokuments ersetzt werden. Soll zudem noch eine Anmerkung notiert werden, kann statt eines Lesezeichens eine persönliche Notiz angefertigt werden. Die Galerie des Hyperaudio-Dokuments repräsentiert gleichermaßen das Inhaltsverzeichnis einer Kurseinheit sowie die Möglichkeit, durch die Seiten zu blättern.
\todo[inline]{Sie sind hier in keiner Weise kritisch gegenüber Ihrer eigenen Arbeit. Es ist keines Falls eine perfekte Lösung, die Sie hier hingelegt haben. Es wird Ihnen auch niemand vorwerfen, Ihr eigenes Werk kritisch zu hinterfragen. Es ist vielmehr eine Frage der Haltung, den jede Kritik ist der Antrieb für eine Verbesserung und manchmal auch Innovation.}

\begin{itemize}
\item Wie kann der Austausch zwischen Studierenden und Lehrenden umgesetzt werden?
\end{itemize}
Die im Hyperaudio"=Plugin umgesetzte Kommentarfunktion ermöglicht den Austausch zwischen Studierenden und Lehrenden. Jeder Teilnehmer kann Kommentare zu bestimmten Zeitpunkten eines Hyperaudio-Dokuments verfassen oder auf bestehende Kommentare antworten.
\todo[inline]{Technisch ja, praktisch aus?}
\todo[inline]{Und? Ist die Darstellung nun gut? Gibt es Daten, die das nahelegen oder sogar bestätigen?}

Mit dem Moodle-Hyperaudio"=Plugin wurde allen Forschungsfragen begegnet. Dennoch ist die Entwicklung einer Hyperaudio-Lernumgebung noch nicht abgeschlossen.
\todo[inline]{Warum? Was wurde beforscht? 

Hatten Sie sich mit dem Design-Based-Research-Ansatz beschäftigt?}

\section{Ausblick}

Im Folgenden soll nun ein Ausblick für mögliche Weiterentwicklungen der Hyperaudio-Lernumgebung gegeben werden.

In der Evaluation wurde festgestellt, dass noch nicht alle Anforderungen zu voller Zufriedenheit umgesetzt wurden. Zu diesen Themen (Favoritenfunktion, Übersichten, Unterstützung mobiler Endgeräte, Fortsetzen Unterbrochener Wiedergaben und Import- und Export-Funktion) sind Nachbesserungen erforderlich. Darüber hinaus wurden bereits in Abschnitt \ref{sec:Verbesserungsvorschlaege} Verbesserungsvorschläge in Form von User Stories festgehalten.\\
Im Bezug auf \ref{US-Kommentar-Bewertung} wäre die Entwicklung einer Bewertungsfunktion für Kommentare und Antworten denkbar, wie sie beispielsweise auf der Plattform \textit{Stack Exchange}\footnote{https://stackexchange.com} zum Einsatz kommt.\\
Eine weitere Überlegung wäre es das Hyperaudio"=Plugin so zu erweitern, dass auch das Streaming auf Fernsehgeräte, zum Beispiel durch \textit{Google ChromeCast} oder \textit{Apple TV}, unterstützt wird. Hierdurch würde die Nutzung einer großen Bildschirmfläche abseits von Laptops und Desktops ermöglicht.\\
Aber auch im Bereich der kleinen Displays, also der mobilen Endgeräte, besteht noch Potenzial zur Weiterentwicklung. Es wäre eine App vorstellbar, welche neben einem speziell für mobile Geräte ausgelegten Design auch über eine Offline-Funktionalität verfügt. Damit würden die Freiheiten des Studierenden nochmals erhöht, da er beim Lernen nicht mehr abhängig von einer Internetverbindung ist.\\
Neben diesen Punkten, die sich vor allem auf die Rolle der Nutzenden beziehen, wären auch einige Weiterentwicklungen für die Administrierenden denkbar. So wäre auf lange Sicht eine Funktion von Nöten, um unerwünschte Nutzerkommentare (vgl. \cite{reinmann2002analyse}) zu löschen.\\
Für Administrierende wäre ebenfalls eine Möglichkeit zur (halb-)automatischen Konvertierung der klassischen Kurseinheiten in Hyperaudio"=Dokumente reizvoll, zum Beispiel durch den Einsatz von ScreenReadern (vgl. \cite{donker2007gestaltung}). Dies würde die Hürde zur Einführung von Hyperaudio"=Dokumenten durch die Lehrenden enorm senken, da dies mit einem wesentlich geringeren Aufwand verbunden wäre.\\
\todo[inline]{So hätte man das vor 10 Jahren gemacht. Welche Möglichkeiten gibt es heute noch?}
Eine weitere Fragestellung, die sich im Rahmen dieser Arbeit eröffnet hat, ist der Umgang mit Änderungen am Hyperaudio"=Dokument. Besonders eine Änderung der zugrundeliegenden Audio-Datei zieht weitreichende Folgen nach sich. So müssen nicht nur Zusatzinhalte und deren Annotationszeitpunkte angepasst, sondern auch bestehende Kommentare, Notizen und Lesezeichen berücksichtigt werden. In diesem Rahmen ist über eine Versionierung beziehungsweise verschiedene Optionen zum Entfernen oder Migrieren der annotierten Inhalte nachzudenken.

Diese Arbeit hat gezeigt, wie eine Hyperaudio-Lernumgebung in Moodle gestaltet werden kann. Das ausbaufähige Hyperaudio"=Plugin bietet Raum für zusätzliche Optimierungen für Nutzende als auch Administrierende.