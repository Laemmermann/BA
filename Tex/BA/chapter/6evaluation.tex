\label{cap:evaluation}
Nachdem die Implementierung des Hyperaudio-Plugins abgeschlossen ist, kann nun die Evaluation anhand der definierten Anforderungen und den dazugehörigen User Stories erfolgen. In diesem Zug werden analog zur Anforderungsdefinition die Bereiche unter Rücksichtnahme der Rollen Administrierende und Nutzende aufgeteilt. Innerhalb dieser beiden Bereiche wird dann analysiert in welchem Umfang die Anforderungen und User Stories durch die Implementierung erfüllt wurden. Die Reihenfolge der Analyse erfolgt unter Zuhilfenahme der festgelegten Prioritäten.

\section{Administration}
\label{sec:eval_administration}
Die Anforderungen der Administrierenden wurden in Tabelle \ref{tab:AnforderungenAdministrierenden} festgehalten.


\textbf{Erstellen eines Hyperaudio-Dokuments}\\
\ref{US-Admin-Erstellen} - erfüllt\\
Die Erstellung von Hyperaudio-Dokumenten, wie es in \ref{US-Admin-Erstellen} gefordert wurde, ist durch das Hochladen einer Audio-Datei, der Zusatzinhalte sowie einer Konfigurationsdatei möglich. Jedoch zeigt diese Implementierung einige Schwächen auf. So fehlt eine Validierung der Konfigurationsdatei. Dies kann zum einen dazu führen, dass das Speichern des Hyperaudio-Dokuments fehlschlägt, wenn beispielsweise der Dateiname eines Zusatzinhaltes innerhalb der JSON-Datei falsch geschrieben wurde. Auch wird nicht sichergestellt, dass sich die Zeiträume der Zusatzinhalte nicht überschneiden. In diesem Fall werden beide Inhalte während der Überschneidung dargestellt und zerstören dabei den Aufbau der Seite. Generell wäre es wünschenswert, wenn die Erstellung von Hyperaudio-Dokumenten komplett innerhalb von Moodle über einen Wizzard erfolgen würde. Weitere Einschränkungen bestehen darin, dass nur eine Audio-Datei als Grundlage für das Hyperaudio-Dokument dienen kann und nur Bilddateien als Zusatzinhalt nutzbar sind.

\textbf{Bearbeiten eines Hyperaudio-Dokuments}\\
\ref{US-Admin-Bearbeiten} - erfüllt\\
Diese Anforderung wurde insofern umgesetzt, als dass es möglich ist im Nachhinein die Audio-Datei, Zusatzinhalte oder Konfigurationsdatei auszutauschen und hierüber Änderungen am Hyperaudio-Dokument vorzunehmen. Ein direkte Bearbeitung des Hyperaudio-Dokuments innerhalb von Moodle kann beim Namen und der Beschreibung erfolgen. Insofern ist eine Bearbeitung von Hyperaudio-Dokumenten möglich, diese beschränkt sich aber darauf, dass die dazugehörigen Dateien ausgetauscht werden können. Dementsprechend ist die Anforderung aus \ref{US-Admin-Bearbeiten} erfüllt, aber es treten die gleichen Nachteile wie bereits beim Erstellen auf.

\textbf{Löschen eines Hyperaudio-Dokuments}\\
\ref{US-Admin-Loeschen} - erfüllt\\
Das Löschen von Hyperaudio-Dokumenten wurde vollständig implementiert

\textbf{Übernahme eines Hyperaudio-Dokuments in einen anderen Kurs}\\
\ref{US-Admin-Kurs} - teilweise erfüllt\\
\ref{US-Admin-Semester} - teilweise erfüllt\\
Die Export- und Import-Funktion ist nur mit den Moodle Boardmitteln möglich. Es ist nur möglich Audio-Datei, Zusatzinhalte und Konfigurationsdatei über die Bearbeitungsmaske herunterzuladen und dann bei der Anlage eines neuen Hyperaudio-Dokuments wiederzuverwenden. Titel und Beschreibung des Hyperaudio-Dokuments müssen händisch nachgetragen werden. Es wäre denkbar für die Zukunft eine richtige Export- und Importfunktion zu implementieren, bei welcher alle Dateien in ein Archiv gepackt werden und um eine weitere Datei für Titel und Beschreibung ergänzt werden. Dieses Archiv müsste sich dann wiederum importieren lassen können, wodurch das ganze Hyperaudio-Dokument angelegt ist.

\textbf{Statistische Auswertungen über die Nutzung der Hyperaudio-Dokumente}\\
\ref{US-Admin-Statistik} - teilweise erfüllt\\
Die gewünschten Statistiken wurden in kleinem Umfang über die \textbf{index.php} bereitgestellt. Mit deren Hilfe ist es Lehrenden möglich zu sehen wie viele Kommentare/Antworten, Notizen und Lesezeichen durch wie viele verschiedene Nutzer erstellt wurden. Es ist aber durchaus vorstellbar, dass diese Statistiken noch erweitert werden. Eine Überlegung wäre, dass visualisiert wird, welche Bereiche des Hyperaudio-Dokuments am häufigsten abgespielt wurden.

 
\section{Nutzung}
In Tabelle \ref{tab:AnforderungenNutzenden} sind die Anforderungen der Nutzenden definiert.

\textbf{Wiedergabe von Hyperaudio-Dokumenten}\\
\ref{US-Wiedergabe} - erfüllt\\
\ref{US-Zeit} - erfüllt (im Bezug auf das alleinige Abspielen)\\
\ref{US-Mobil} - erfüllt (im Bezug auf das alleinige Abspielen)\\
\ref{US-Zeit-Mobil} - erfüllt (im Bezug auf das alleinige Abspielen)\\
Durch die Implementierung wurde dieses Ziel erreicht. Es kann eine Audio-Datei abgespielt werden und gleichzeitig werden zu den festgelegten Zeitpunkten die annotierten Zusatzinhalte dargestellt. Hierbei liegt die bereits in Abschnitt \ref{sec:eval_administration} erwähnte Einschränkung bezüglich der Formate der Zusatzinhalte vor. Zusätzlich gibt es keine Möglichkeit die Hyperaudio-Dokumente in einen Vollbildmodus abzuspielen.

\textbf{Hinweise auf die Darstellung von annotierten Zusatzinhalten}\\
\ref{US-Zeit} - erfüllt(im Bezug auf das Abspielen der Audio Cues)\\
\ref{US-Mobil} - erfüllt(im Bezug auf das Abspielen der Audio Cues)\\
\ref{US-Zeit-Mobil} - erfüllt(im Bezug auf das Abspielen der Audio Cues)\\
Gleichzeitig mit der Darstellung von Zusatzinhalten wird eine Audio Cue abgespielt. Wie bereits in Abschnitt \ref{sec:audiocues} erwähnt, wäre es möglich diese Funktion noch um weitere Töne zu erweitern, um auf verschiedene Arten von Zusatzinhalten hinzuweisen.

\textbf{Übersicht über annotierte Zusatzinhalte}\\
\ref{US-Galerie} - erfüllt\\
Der Übersicht über annotierte Zusatzinhalte wurde durch die Implementierung einer Galerie für Zusatzinhalte Sorge getragen. Durch die Erweiterung der Galerie um zusätzliche Metadaten und einer Rückkopplung an den Player wurde die Funktionalität der Galerie zusätzlich über die eigentliche Anforderung hinaus erweitert.


\textbf{Kommentarfunktion bei Hyperaudio-Dokumenten}\\
\ref{US-Kommentar-L} - erfüllt\\
\ref{US-Antwort-L} - erfüllt\\
\ref{US-Kommentar-S} - erfüllt\\
\ref{US-Antwort-S} - erfüllt\\
\ref{US-Sortierung-Zeitpunkt} - erfüllt (im Bezug auf das alleinige Darstellen von Kommentaren)\\
\ref{US-Suche} - erfüllt\\
Den Anforderungen an eine Kommentarfunktion wurde Genüge getan, da das Erstellen, Anzeigen und Beantworten von Kommentaren möglich ist. Die Suchfunktion wurde für Kommentare als auch Notizen implementiert. Somit sind die Anforderungen an die Kommentarfunktion zwar erfüllt, aber ein verbesserter Suchalgorithmus, welcher Schreibfehler verzeiht und ähnliche Wörter berücksichtigt, würde die Suche noch aufwerten. Ebenfalls wäre es bei Kommentaren und deren Antworten noch möglich, dass ab einer gewissen Anzahl an Antworten nicht mehr alle dargestellt werden. Auch ein Bewertungssystem für Kommentare und Antworten, wie es zum Beispiel bei Stack Overflow umgesetzt ist wäre sicherlich eine sinnvolle Erweiterung. 


\textbf{Notizfunktion bei Hyperaudio-Dokumenten}\\
\ref{US-Notiz-L} - erfüllt\\
\ref{US-Notiz-S} - erfüllt\\
\ref{US-Notiz-Bearbeiten} - erfüllt\\
\ref{US-Notiz-Loeschen} - erfüllt\\
\ref{US-Sortierung-Zeitpunkt} - erfüllt (im Bezug auf das alleinige Darstellen von Notizen)\\
Die gewünschten Möglichkeiten wurden umgesetzt. Wie bereits bei der Kommentarfunktion erwähnt, wurde innerhalb der Notizen eine Suche ermöglicht. Es wäre aber vorstellbar, dass Notizen zukünftig auch um eigene Dateien ergänzt werden können, um beispielsweise selbst angefertigte Zeichnungen bei Notizen zu hinterlegen. 
\textbf{Lesezeichenfunktion bei Hyperaudio-Dokumenten}\\
\ref{US-Lesezeichen} - erfüllt\\
\ref{US-Lesezeichen-Loeschen} - erfüllt\\
Nutzende sind in der Lage Lesezeichen zu erstellen und zu löschen. Eine Weiterentwicklungsmöglichkeit wäre, dass Markierungen nicht nur punktuell sondern für bestimmte Zeiträume definiert werden können.

\textbf{Filter- und Sortiermöglichkeiten}\\
\ref{US-Filter} - erfüllt\\
\ref{US-Sortierung-Erstellungsdatum} - erfüllt\\
\ref{US-Sortierung-Zeitpunkt} - erfüllt ( in Bezug auf die Sortierfunktion)\\
Durch die Implementierung ist es möglich zu filtern, ob Kommentare, Notizen oder beides dargestellt werden sollen. Auch eine Sortierung nach Erstelldatum (auf- und absteigend) sowie nach Annotationszeitpunkt ist umgesetzt worden.

\textbf{Favoritenfunktion für Hyperaudio-Dokumente}\\
\ref{US-Favoriten} - nicht erfüllt\\
\ref{US-Favoriten-Loeschen} - nicht erfüllt\\

\textbf{Übersicht über alle Hyperaudio-Dokumente der belegten Kurse}\\
\ref{US-Uebersicht-Kurse} - teilweise erfüllt\\
Nutzende können sich nur innerhalb eines Kurses einen Überblick über alle darin enthaltenen Hyperaudio-Dokumente verschaffen.


\textbf{Übersicht über die zuletzt abgespielten Hyperaudio-Dokumente}\\
\ref{US-Uebersicht-Letzte} - nicht erfüllt\\


\textbf{Funktion zum Fortsetzen unterbrochener Wiedergaben bei folgenden Aufrufen in Moodle}\\
\ref{US-Fortsetzen} - nicht erfüllt\\


\textbf{Unterstützung von mobilen Endgeräten}\\
\ref{US-Mobil} - teilweise erfüllt\\
\ref{US-Zeit-Mobil}  - teilweise erfüllt\\
Es wurde eine angepasste Darstellung auf mobilen Geräten implementiert. Diese gibt dem Nutzer die Möglichkeit auch auf mobilen Endgeräten Hyperaudio-Dokumente abzuspielen, die dazugehörigen Kommentare, Notizen und Lesezeichen als auch die Galerie zu betrachten. Hierbei handelt es sich aber nur um eine angepasste Darstellung der Desktopversion. Manche Funktionen wie die Interaktion mit den Blöcken innerhalb der Mediensteuerung und das Erstellen und Löschen von Markierungen sind allerdings nicht funktional. Um die Nutzung auf mobilen Geräten komfortabel zu gestalten und alle Funktionen anbieten zu können, wäre eine komplett angepasste Darstellung der Hyperaudio-Dokumente notwendig.


\section{Zusammenfassung}

Durch die vorliegende Implementierung des Hyperaudio-Plugins wurden bereits viele der definierten Anforderungen erfüllt, dennoch gibt es sowohl im Bereich der Administration als auch im Bereich der Nutzung noch Verbesserungspotenzial.