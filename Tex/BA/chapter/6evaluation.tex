\label{cap:evaluation}
Nachdem nun die Implementierung des Hyperaudio-Plugins abgeschlossen ist, kann die Evaluation anhand der im Abschnitt \ref{sec:UserStories} formulierten User Stories und den daraus abgeleiteten Anforderungen aus Abschnitt \ref{sec:anforderungsdefinition} erfolgen. Daraufhin werden mögliche Verbesserungen auf Basis von neuen User Stories formuliert.
\todo[inline]{überarbeiten}

\section{User Stories}
Nachfolgend wird untersucht, ob die User Stories mit der vorhandenen Implementierung umgesetzt werden können.

\begin{enumerate}[label=US-\arabic*:,ref=US-\arabic*]

\item \label{US-Admin-Erstellen-Eval} \textit{Als Administrierende möchte Prof. Dr. Karolin Schröder ein neues Hyperaudio-Dokument in ihrem Kurs \glqq Einführung in die Wirtschaftsinformatik\grqq{} zur Verfügung stellen, um den Studierenden neue Lerninhalte bereitzustellen.}\\
Prof. Dr. Karolin Schröder ist in der Lage ein neues Hyperaudio-Dokument zu erstellen, indem sie Ihrem Kurs eine neue Aktivität \textit{Hyperaudio} hinzufügt. Hierbei muss sie eine Audio-Datei, die gewünschten Zusatzinhalte und eine Konfigurationsdatei bereitstellen.


\item \label{US-Admin-Loeschen-Eval} \textit{Als Administrierende möchte Prof. Dr. Karolin Schröder Hyperaudio-Dokumente aus ihrem Kurs \glqq Einführun in die Wirtschaftsinformatik\grqq{} löschen können, um veraltete Informationen zu entfernen.}\\
Das Hyperaudio-Dokument kann Prof. Dr. Karolin Schröder löschen, indem sie die dazugehörige Aktivität \textit{Hyperaudio} entfernt.

\item \label{US-Admin-Semester-Eval} \textit{Als Administrierende möchte Prof. Dr. Karolin Schröder Hyperaudio-Dokumente aus ihrem Kurs \glqq Einführung in die Wirtschaftsinformatik\grqq{} im Sommersemester in den darauffolgenden Kurs im Wintersemester übernehmen, um diese nicht erneut erstellen zu müssen.}\\
Die Übernahme von Hyperaudio-Dokumenten in den darauffolgenden Kurs im Wintersemester ist Prof. Dr. Karolin Schröder nur über Umwege möglich. Hierzu müssen die einzelnen Dateien der Hyperaduio-Dokumente heruntergelanden werden und dann bei der Erstellung im neuen Kurs erneut hochgelanden werden. Der Titel und die Beschreibung des Hyperaudio-Dokuments muss manuell übernommen werden.

\item \label{US-Admin-Kurs-Eval} \textit{Als Administrierende möchte Prof. Dr. Karolin Schröder Hyperaudio-Dokumente anderer Kurse in ihren Kurs \glqq Einführung in die Wirtschaftsinformatik\grqq{} übernehmen, um auf die hervorragende Arbeit anderer Lehrender zurückgreifen zu können, da sich die Themen mit ihrem Kurs überschneiden.}\\
Die Hyperaudio-Dokumente anderer Kurse können Prof. Dr. Karolin Schröder analog zum Vorgehen bei \ref{US-Admin-Semester-Eval} nur anhand der einzelnen Dateien der Hyperaudio-Dokumente bereitgestellt werden. Diese müssen dann beim Erstellen der Hyperaudio-Dokumente verwendet werden.

\item \label{US-Admin-Statistik-Eval} \textit{Als Administrierende möchte Prof. Dr. Karolin Schröder Erkenntnisse daraus gewinnen, wie die Hyperaudio-Dokumente des Kurses \glqq Einführung in die Wirtschaftsinformatik\grqq{} von Studierenden genutzt werden, um Verbesserungspotenzial auszumachen.}\\
Prof. Dr. Karolin Schröder ist es möglich, über den Block \textit{Aktivitäten} zu der Auflistung aller Hyperaudio-Aktivitäten ihres Kurses zu gelangen. In dieser Auflistung kann sie sowohl die Anzahl der erstellten öffentlichen Kommentare, persönlichen Notizen und Lesezeichen aller Nutzer entnehmen, als auch die Anzahl der Nutzer, die diese geniert haben.

\item \label{US-Admin-Bearbeiten-Eval} \textit{Als Administrierender möchte Dr. Julian Schmidt ein vorhandenes Hyperaudio-Dokument in dem von ihm betreuten Kurs \glqq Marketing\grqq{} überarbeiten, um einen Fehler zu beseitigen.}\\
Das Bearbeiten von Hyperaudio-Dokumenten ist durch den Austausch der Audio-Datei oder der Konfigurationsdatei möglich oder durch das Ergänzen, Ändern oder Löschen von Zusatzinhalten, was mit dem Austausch der Konfigurationsdatei verbunden ist.
\todo[inline]{Überarbeiten}

\item \label{US-Wiedergabe-Eval} \textit{Als Nutzende möchte Prof. Dr. Karolin Schröder die bereits vorhandenen Hyperaudio-Dokumente aus ihrem Kurs \glqq Einführung in die Wirtschaftsinformatik\grqq{} wiedergeben, um diese auf ihre Richtigkeit zu überprüfen.}\\
Das Hyperaudio-Dokument kann nach dem Öffnen der \textit{Hyperaudio}-Aktivität wiedergegeben werden.

\item \label{US-Antwort-L-Eval} \textit{Als Nutzende möchte Prof. Dr. Karolin Schröder die Kommentare zu einem Hyperaudio-Dokument lesen und beantworten können, um auf Fragen von Studierenden einzugehen.}\\
Nach dem Öffnen einer \textit{Hyperaudio}-Aktivität sind alle an das Hyperaudio-Dokument annotierten Kommentare sichtbar. Diese können nach einem Klick auf \glqq Antworten\grqq{} beantwortet werden.

\item \label{US-Notiz-L-Eval} \textit{Als Nutzende möchte Prof. Dr. Karolin Schröder eine Notiz zu einem Hyperaudio-Dokument machen, um ihren Gedanken festzuhalten und später darauf zurückgreifen zu können.}\\
Das Erstellen einer Notiz wird durch das Befüllen des entsprechenden Textfeldes und der Betätigung des \glqq Als Notiz speichern\grqq{}-Buttons erreicht. Die Notiz wird zum aktuellen Zeitpunkt der Wiedergabe annotiert.

\item \label{US-Kommentar-L-Eval} \textit{Als Nutzender möchte Dr. Julian Schmidt eine gefundene Erklärungslücke in einem Hyperaudio-Dokument durch einen Kommentar zum entsprechenden Zeitpunkt schließen, um eventuellen Fragen der Studierenden zuvorzukommen.}\\
Dr. Julian Schmidt kann durch Befüllen des entsprechenden Textfeldes und der Betätigung des \glqq Als Kommentar speichern\grqq{}-Buttons einen Kommentar hinterlegen. Der Kommentar wird zum aktuellen Zeitpunkt der Wiedergabe annotiert.

\item \label{US-Zeit-Eval} \textit{Als Nutzende möchte Laura Ebert mittels Hyperaudio-Dokument lernen, um die Zeit während Haushaltsarbeiten, wie dem Bügeln, Kochen oder Putzen, und dem Pendeln sinnvoller zu nutzen.}\\
Durch die auditive Wiedergabeform der Hyperaudio-Dokumente ist Laura Ebert in der Lage in den gewünschten Situationen mit Hyperaudio-Dokumenten zu lernen. Falls ein visueller Zusatzinhalt dargestellt wird, so wird sie durch die Wiedergabe einer Audio Cue auf diesen hingewiesen.

\item \label{US-Uebersicht-Kurse-Eval} \textit{Als Nutzende möchte Laura Ebert erfahren, welche Hyperaudio-Dokumente in den von ihr belegten Kursen angeboten werden, um herauszufinden, mit welchen Mitteln sie sich auf die anstehenden Prüfungen vorbereiten kann.}\\
Laura Ebert kann sich mithilfe des Blocks \textit{Aktivitäten} ausschließlich einen Überblick über alle Hyperaudio-Dokumente eines jeweiligen Kurses verschaffen. Eine kursübergreifende Übersicht ist nicht vorhanden.

\item \label{US-Kommentar-S-Eval} \textit{Als Nutzende möchte Laura Ebert einen Kommentar verfassen, um dem Kursbetreuer und den anderen Studierenden eine Frage zu stellen.}\\
Indem sie ihre Frage in das Kommentarfeld schreibt und dann mittels des entsprechenden Buttons speichert, kann Laura Ebert eine Frage an den Kursbetreuer und andere Studierende stellen.

\item \label{US-Lesezeichen-Eval} \textit{Als Nutzende möchte Laura Ebert ein Lesezeichen setzen, wenn eine klausurrelevante Thematik erklärt wird. Bei der Prüfungsvorbereitung möchte sie anhand dieser Lesezeichen diejenigen Themen erkennen, mit welchen sie sich besonders intensiv beschäftigen möchte.}\\
Das Hinterlegen von Lesezeichen ist durch die Betätigung des Lesezeichen-Buttons links neben der Timeline möglich. Das Lesezeichen wird zum Wiedergabezeitpunkt erstellt und in Form eines Lesezeichen-Symbols in der Timeline dargestellt.

\item \label{US-Lesezeichen-Loeschen-Eval} \textit{Als Nutzende möchte Laura Ebert ein Lesezeichen löschen, da sie den markierten Lerninhalt inzwischen beherrscht. Anhand der übrigen Lesezeichen möchte sie schnell erkennen, wo für sie noch Lernbedarf besteht.}\\
Lesezeichen können durch Rechtsklick auf das entsprechende Lesezeichen-Symbol in der Timeline gelöscht werden.

\item \label{US-Notiz-S-Eval} \textit{Als Nutzende möchte Laura Ebert eine Notiz erstellen, um ein Beispiel zu dem genannten Sachverhalt festzuhalten, sodass sie die Thematik beim nächsten Mal einfacher nachvollziehen kann.}\\
Das Erstellen einer Notiz wird durch das Befüllen des entsprechenden Textfeldes und der Betätigung des \glqq Als Notiz speichern\grqq{}-Buttons erreicht. Die Notiz wird zum aktuellen Zeitpunkt der Wiedergabe annotiert.

\item \label{US-Fortsetzen-Eval} \textit{Als Nutzende möchte Laura Ebert die Wiedergabe eines Hyperaudio-Dokuments beenden und am nächsten Tag automatisch an derselben Stelle fortsetzen können, um das Lernen schnell wiederaufnehmen zu können.}\\
Eine automatische Wiedergabe des Hyperaudio-Dokuments an der Stelle, an der es beendet wurde, ist nicht möglich.

\item \label{US-Mobil-Eval} \textit{Als Nutzende möchte Laura Ebert die Hyperaudio-Angebote mit ihrem Smartphone in Anspruch nehmen, um auch die Zeit während des Pendelns zum Lernen nutzen zu können.}\\
Durch die angepasste Darstellung auf mobilen Endgeräten kann Laura Ebert auch mit ihrem Smartphone das Hyperaudio-Angebot wahrnehmen. Dabei ergibt sich allerdings die Einschränkung, dass sie vorhandene Lesezeichen nicht löschen kann.

\item \label{US-Notiz-Bearbeiten-Eval} \textit{Als Nutzender möchte Max Lustig eine alte Notiz bearbeiten, um einen Schreibfehler zu korrigieren.}\\
Durch Betätigen von \glqq Bearbeiten\grqq{}-Schaltfläche ist Max Lustig in der Lage seine Notiz zu bearbeiten und die Änderung zu speichern.

\item \label{US-Notiz-Loeschen-Eval} \textit{Als Nutzender möchte Max Lustig eine alte Notiz löschen, da er inzwischen Lernfortschritte gemacht hat und auf diese Notiz verzichten kann.}\\
Durch Betätigen von \glqq Löschen\grqq{}-Schaltfläche kann die Notiz durch Max Lustig gelöscht werden.


\item \label{US-Galerie-Eval} \textit{Als Nutzender möchte Max Lustig schnell erkennen welche Inhalte im Hyperaudio-Dokument behandelt werden, um eine Erklärung eines bestimmten Themas zu finden.}\\
Durch Betrachtung der annotierten Zusatzinhalte, welche gesammelt in der Galerie dargestellt werden, kann sich Max Lustig schnell einen groben Überblick über das Hyperaudio-Dokument verschaffen.

\item \label{US-Suche-Eval} \textit{Als Nutzender möchte Max Lustig nach Textinhalten in Kommentaren suchen können, um schnell Erklärungen zu finden.}\\
Durch Nutzung des Suchfeldes können sowohl Kommentare als auch Notizen durchsucht werden.

\item \label{US-Sortierung-Erstellungsdatum-Eval} \textit{Als Nutzender möchte Max Lustig die Kommentare nach Erstellungsdatum sortieren können, um sich einen Überblick über die neuesten Aktionen zu verschaffen.}\\
Wenn Max Lustig im Dropdown-Menü für die Sortierung \glqq Erstelldatum absteigend\grqq{} auswählt, kann er sich einen Überblick über die neuesten Aktionen verschaffen.

\item \label{US-Sortierung-Zeitpunkt-Eval} \textit{Als Nutzender möchte Max Lustig die Kommentare und persönlichen Notizen zu den Annotationszeitpunkten zuordnen können, um diese bei der Wiedergabe verfolgen zu können.}\\
Wenn Max Lustig im Dropdown-Menü für die Sortierung \glqq Annotationszeitpunkt\grqq{} auswählt, werden die Kommentare und Notizen entsprechend sortiert. Zusätzlich werden die Kommentare und Notizen in der Timeline visualisiert.

\item \label{US-Filter-Eval} \textit{Als Nutzender möchte Max Lustig öffentliche Kommentare und persönliche Notizen getrennt betrachten können, um die öffentliche Diskussion verfolgen beziehungsweise die eigenen Anmerkungen isoliert betrachten zu können.}\\
Mittels zweier Checkboxen kann festgelegt werden, ob Kommentare und/oder Notizen dargestellt werden sollen.

\item \label{US-Antwort-S-Eval} \textit{Als Nutzender möchte Max Lustig auf Kommentare antworten können, um sich mit den Studierenden und Lehrenden auszutauschen.}\\
Nach dem Öffnen einer \textit{Hyperaudio}-Aktivität sind alle an das Hyperaudio-Dokument annotierten Kommentare sichtbar. Diese können nach einem Klick auf \glqq Antworten\grqq{} beantwortet werden.

\item \label{US-Uebersicht-Letzte-Eval} \textit{Als Nutzender möchte Max Lustig erkennen, welche Hyperaudio-Dokumente er zuletzt abgespielt hat, um seinen Lernfortschritt im Auge zu behalten.}\\
Eine Übersicht über die zuletzt abgespielten Hyperaudio-Dokumente ist nicht vorhanden.

\item \label{US-Favoriten-Eval} \textit{Als Nutzender möchte Max Lustig besonders hilfreiche Hyperaudio-Dokumente als Favoriten speichern, um diese schnell als solche identifizieren zu können.}\\
Das Erstellen von Favoriten ist nicht möglich.

\item \label{US-Favoriten-Loeschen-Eval} \textit{Als Nutzender möchte Max Lustig die Markierung als Favorit entfernen können, wenn der Inhalt für ihn nicht mehr von Interesse ist.}\\
Das Löschen von Favoriten ist nicht möglich.

\item \label{US-Zeit-Mobil-Eval} \textit{Als Nutzender möchte Max Lustig auf seinem Tablet Zugang zu Hyperaudio-Dokumenten haben, um die Zeit auf dem Laufband gleichzeitig zum Lernen nutzen zu können.}\\
Durch die angepasste Darstellung auf mobilen Endgeräten kann Max Lustig auch mit seinem Tablet das Hyperaudio-Angebot wahrnehmen. Dabei hat er allerdings die Einschränkung, dass er vorhandene Lesezeichen nicht löschen kann.
\end{enumerate}

Wie zu erkennen ist, wurde mit 22 User Stories ein Großteil der User Stories umgesetzt. Jeweils vier User Stories wurden nur teilweise oder gar nicht bei der Implementierung berücksichtigt.


\section{Anforderungen}
Nachdem im vorherigen Abschnitt das Hyperaudio-Plugin auf Basis der User Stories bewertet wurde, werden die Ergebnisse nun auf die Anforderungen aus Kapitel \ref{sec:anforderungsdefinition} übertragen. Die Resultate sind in Tabelle \ref{tab:EvalAnforderungenAdministrierenden} und \ref{tab:EvalAnforderungenNutzenden} aufgeführt.


\begin{table}[!ht]
\def\arraystretch{1.4}
\caption{Evaluierung der Anforderungen der Administrierenden}
\label{tab:EvalAnforderungenAdministrierenden}
 \begin{tabularx}{\textwidth}{lXcc}      
    \hline
    Nr. & Anforderung & Priorität & Erfüllungsgrad
    \\\hline
    1 & Erstellen eines Hyperaudio-Dokuments & hoch & erfüllt\\
    2 & Bearbeiten eines Hyperaudio-Dokuments & hoch & erfüllt\\
    3 & Löschen eines Hyperaudio-Dokuments & hoch & erfüllt\\
    4 & Übernahme eines Hyperaudio-Dokuments in einen anderen Kurs & mittel & teilweise erfüllt\\
    5 & Statistische Auswertungen über die Nutzung der Hyperaudio-Dokumente & niedrig & erfüllt\\
    \hline
    \end{tabularx}
\end{table}

\begin{table}[!ht]
\def\arraystretch{1.4}
\caption{Evaluierung der Anforderungen der Nutzenden}
\label{tab:EvalAnforderungenNutzenden}
\begin{tabularx}{\textwidth}{lXcc}      
    \hline
    Nr. & Anforderung & Priorität & Erfüllungsgrad
    \\\hline
    1 & Wiedergabe von Hyperaudio-Dokumenten & hoch & erfüllt\\
    2 & Hinweise auf die Darstellung von annotierten Zusatzinhalten & hoch & erfüllt\\
    3 & Übersicht über annotierte Zusatzinhalte & mittel & erfüllt\\
    4 & Kommentarfunktion bei Hyperaudio-Dokumenten & & \\
    4.1 & \hspace*{0.5cm} Erstellen von Kommentaren & hoch & erfüllt\\
    4.2 & \hspace*{0.5cm} Anzeigen von Kommentaren & hoch & erfüllt\\
    4.3 & \hspace*{0.5cm} Antworten auf Kommentare & hoch & erfüllt\\
    4.4 & \hspace*{0.5cm} Suchfunktion innerhalb der Kommentare & mittel & erfüllt\\ 
    5 & Notizfunktion bei Hyperaudio-Dokumenten & & \\
    5.1 & \hspace*{0.5cm} Erstellen von Notizen & hoch & erfüllt\\
    5.2 & \hspace*{0.5cm} Anzeigen von Notizen & hoch & erfüllt\\
    5.3 & \hspace*{0.5cm} Bearbeiten von Notizen & hoch & erfüllt\\
   	5.4 & \hspace*{0.5cm} Löschen von Notizen & hoch & erfüllt\\
    6 & Lesezeichenfunktion bei Hyperaudio-Dokumenten & & \\
    6.1 & \hspace*{0.5cm} Erstellen von Lesezeichen & mittel & erfüllt\\
    6.2 & \hspace*{0.5cm} Anzeigen von Lesezeichen & mittel & erfüllt\\
   	6.3 & \hspace*{0.5cm} Löschen von Lesezeichen & mittel & erfüllt\\
   	7 & Filter- und Sortiermöglichkeiten & mittel & erfüllt\\
    8 & Favoritenfunktion für Hyperaudio-Dokumente & & \\
    8.1 & \hspace*{0.5cm} Erstellen von Favoriten & niedrig & nicht erfüllt\\
    8.2 & \hspace*{0.5cm} Anzeigen von Favoriten & niedrig & nicht erfüllt\\
    8.3 & \hspace*{0.5cm} Löschen von Favoriten & niedrig & nicht erfüllt\\    
    9 & Übersicht über alle Hyperaudio-Dokumente der belegten Kurse & niedrig & nicht erfüllt\\
    10 & Übersicht über die zuletzt abgespielten Hyperaudio-Dokumente & niedrig & nicht erfüllt\\
    11 &  Funktion zum Fortsetzen unterbrochener Wiedergaben bei folgenden Aufrufen in Moodle & niedrig & nicht erfüllt\\
    12 & Unterstützung von mobilen Endgeräten & mittel & teilweise erfüllt\\
    \hline
\end{tabularx}
\end{table}
\FloatBarrier
\section{Verbesserungsvorschläge}
Dies Verbesserungsvorschläge für das Hyperaudio-Plugin, die bei der Implementierung und Evaluation entstanden sind,  sollen in Form von neuen User Stories festgehalten werden. Für diese User Stories werden die gleichen Personas aus Abschnitt \ref{sec:personas} verwendet wie bereits bei den User Stories aus Abschnitt \ref{sec:UserStories}.

\begin{enumerate}[label=US-\arabic*:,ref=US-\arabic*]
\setcounter{enumi}{30}
\item Als Administrierende möchte Prof. Dr. Karolin Schröder möchte detailliert auf Fehler in der Konfigurationsdatei hingewiesen werden, um Fehler in der Konfigurationsdatei schneller zu finden, wie beispielsweise Zeitüberschneidungen bei annotierten Zusatzinhalten.
\item Als Administrierende möchte Prof. Dr. Karolin Schröder Hyperaudio-Dokumente komplett innerhalb der Moodle-Umgebung erzeugen können, um nicht mühsam eine Konfigurationsdatei erstellen zu müssen.
\item Als Administrierende möchte Prof. Dr. Karolin Schröder mehrere Audio-Dateien für Hyperaudio-Dokumente verwenden können, um die Vertonung von Kurseinheiten in mehreren Abschnitten durchführen zu können.
\item Als Administrierende möchte Prof. Dr. Karolin Schröder auch andere Dateiformate als Bilddateien als Zusatzinhalt annotieren können, um auch PDF-Dokumente oder Videos annotieren zu können.
\item Als Administrierende möchte Prof. Dr. Karolin Schröder noch detaillierte Auswertungen über das Nutzungsverhalten von Hyperaudio-Dokumenten erhalten, um noch genauere Rückschlüsse über die Stärken und Schwächen des Hyperaudio-Dokuments zu ermitteln. 
\item Als Administrierende möchte Prof. Dr. Karolin Schröder verschiedene Audio Cues verwenden können, um die Studierende auf die verschiedenen Arten von Zusatzinhalten hinzuweisen.
\item Als Nutzende möchte Laura Ebert zunächst nur noch die wichtigsten Antworten auf Kommentare angezeigt bekommen, um eine bessere Übersicht bei viel diskutierten Hyperaudio-Dokumenten zu erhalten.
\item Also Nutzender möchte Laura Ebert ihrer Notiz eine Datei hinzufügen, um ihre selbst angefertigte Zeichnung zu hinterlegen.
\item Als Nutzender möchte Max Lustig eine bessere Suchfunktion, um auch Ergebnisse trotz Schreibfehler oder mit ähnlichen Wörtern zu finden.
\item Als Nutzender möchte Max Lustig nicht nur Zeitpunkte sondern auch Zeitfenster hervorheben, um genau zu wissen welches Zeitfenster wichtige Inhalte enthält.
\item Als Nutzender möchte Max Lustig alles Funktionalitäten der Desktop-Version auf seine Smartphone nutzen können, um unabhängiger von seinem Desktop-Rechner zu werden.
\end{enumerate}

Diese neu formulierten User Stories stellen erneut nur einen Teil der möglichen User Stories dar und können für die Optimierung des Hyperaudio-Plugins dienen.

\section{Zusammenfassung}
Bei der Evaluation wurde die vorliegende Implementierung des Hyperaudio-Plugins analysiert. Die Ergebnisse zeigen auf, dass bereits viele der formulierten User Stories ermöglicht und dementsprechend viele Anforderungen erfüllt wurden. Dennoch gibt es sowohl im Bereich der Administration als auch im Bereich der Nutzung noch Verbesserungspotenzial, diese wurden in Form von neuen User Stories formuliert.