Mit den Erkenntnissen des vorherigen Kapitels können wir uns nun der Konzeption unseres Moodle-Plugins zuwenden. Dabei werden wir zu Beginn verschiedene Nutzungsszenarien basierend auf den bereits festgelegten Anforderungen definieren. Diese sollen dann im späteren Verlauf auch zur Evaluierung der Implementierung herangezogen werden. Mit diesen Nutzungsszenarien im Hinterkopf wenden wir uns dann im nächsten Schritt der Benutzeroberfläche und deren Gestaltung zu. Abschließend haben wir ausreichend Vorarbeiten geleistet, um die Architektur des Plugins festzulegen und das Schnittstellenformat zu definieren. Diese stellen dann die letzten Schritte vor der Implementierung des Plugins dar.



%%%%%%%%%%
\section{Gestaltung der Benutzeroberfläche}
Damit den Lehrenden und Studierenden die im vorherigen Abschnitt beschriebenen Nutzungsszenarien möglichst leicht fallen, wenden wir uns nun der Gestaltung der Benutzeroberfläche zu. \glqq Das Design der Benutzeroberfläche stellt einen zentralen Aspekt für die Gebrauchstauglichkeit eines Softwareprodukts dar\grqq{} \citep[S. 1]{oppermann2002user}. Einen dementsprechend hohen Stellenwert wollen wir der Benutzeroberfläche unseres Moodle-Plugins zuschreiben. Bei der Gestaltung der Benutzeroberfäche gehen wir wie bereits bei der Analyse in Kapitel \ref{cha:analyse} vor und teilen die Benutzeroberfläche in Teilbereiche auf. Im ersten Schritt betrachten wir zunächst die Seite eines Hyperaudio-Dokuments innerhalb eines Kurses. Danach widmen wir uns der Administrationsseite eines Hyperaudio-Dokuments innerhalb eines Kurses. Im letzten Schritt wenden wir uns den verschiedenen Integrationsmöglichkeiten innerhalb der allgemeinen Moodle-Oberfläche zu.

Generell erfolgen alle Entscheidungen bezüglich der Oberfläche auf Basis von Skizzen. Diese wurden mittels des Programms \textit{Balsamiq Mockups 3.5.15}\footnote{https://balsamiq.com/} erstellt. Anhand der Skizzen können Vor- und Nachteile der verschiedenen Designansätze schnell erkannt und auf Grund dessen sachliche Entscheidungen getroffen werden. 
%Durch das Arbeiten mit Skizzen erkennt man schneller welche Vor- und Nachteile die verschiedenen Designansätze bieten und kann auf Grund dessen dann sachliche Entscheidungen treffen.


%%%%%%%%%%
%\subsection{Seite eines Hyperaudio-Dokuments}
Die Seite eines Hyperaudio-Dokuments lässt sich grob, wie bereits in Abbildung \ref{fig:MockupBereiche} dargestellt, in die Bereiche Player, Galerie und Kommentarsektion aufteilen. Wir werden nun zunächst für jeden dieser Bereiche verschiedene Designs diskutieren und uns dann für eines entscheiden. Danach erfolgt die Entscheidung über die Anordnung dieser Bereiche auf der Seite eines Hyperaudio-Dokuments.


%%%%%%%%%%
\subsubsection{Player}
Beim Player für Hyperaudio-Dokumente müssen, neben den üblichen Mediensteuerungselementen, gleich mehrere zusätzliche Elemente visualisiert werden. Zum einen müssen zu den entsprechenden Zeitpunkten die annotierten Zusatzinhalte dargestellt werden. Auf der anderen Seiten sollen auch die annotierten öffentlichen Kommentare, persönlichen Notizen und Markierungen veranschaulicht werden. Dem Wunsch, direkt über den Player öffentliche Kommentare, persönlichen Notizen und Markierungen erstellen und in letzterem Fall sogar löschen zu können, muss auch Sorge getragen werden.

Der Player für Hyperaudio-Dokumente wird, wie in Abbildung \ref{fig:MockupPlayerVersion1} zusehen ist, als Videoplayer umgesetzt. Somit werden die Zusatzinhalte an Stelle eines Videos dargestellt. Die persönlichen Notizen und Markierungen werden innerhalb der Abspielleiste mittels unterschiedlich gefärbter Kreisen illustriert. In diesem Fall sollen die roten Kreise Markierungen und der blaue Kreis eine persönliche Notiz widerspiegeln. Unterhalb der Mediensteuerung ist ein Bereich zu finden, in dem die Kommentare grafisch sichtbar gemacht werden sollen. Hierfür wird jedes Hyperaudio-Dokument in die gleiche fixe Anzahl an Zeitfenstern aufgeteilt. Diese Zeitfenster werden durch senkrecht orientierte Balken dargestellt, deren Höhe für die Anzahl der zu diesem Zeitfenster erfassten Kommentare stehen soll. Unter dem Bereich für die Kommentare befindet sich eine Eingabemaske, mit welcher öffentliche Kommentare und persönliche Notizen erfasst werden können. Das Erstellen und Löschen von Markierungen soll mittels Rechtsklick auf die entsprechende Stelle innerhalb der Abspielleiste in einem dazugehörigen Kontextmenü umgesetzt werden. Dies ist in Abbildung \ref{fig:MockupPlayerVersion1} mittels der beiden Mauszeiger, den Pfeilen und den entsprechenden Buttons symbolisiert.

\todo[inline]{Unterscheidung nicht nur durch Farbe (Barrierefreiheit)}

%\begin{figure}[h!]
%\includegraphics[width=0.8\textwidth,center]{MockupPlayerVersion1.pdf}
%\caption{\label{fig:MockupPlayerVersion1} Erster Entwurf des Players}
%\end{figure}

In einer zweiten Variante des Players wird die Visualisierung der persönlichen Notizen von der Abspielleiste in den Bereich der Kommentare verschoben. Wie in Abbildung \ref{fig:MockupPlayerVersion2} ersichtlich,  wird der Balken für den Zeitraum, in dem die persönlichen Notiz liegt, zu einem gewissen Teil blau eingefärbt. Dadurch wird nebenbei das Handling der Punkte in der Abspielleiste vereinheitlicht, da es hier nur noch die Markierungen mit Interaktionsmöglichkeit gibt.

\begin{figure}[h!]
\begin{subfigure}[c]{\textwidth}
\includegraphics[width=0.8\textwidth,center]{MockupPlayerVersion1.pdf}
\subcaption{Erste Version}
\label{fig:MockupPlayerVersion1}
\end{subfigure}
\par\bigskip
\begin{subfigure}[c]{\textwidth}
\includegraphics[width=0.8\textwidth,center]{MockupPlayerVersion2.pdf}
\subcaption{Finale Version}
\label{fig:MockupPlayerVersion2}
\end{subfigure}
\caption{Benutzeroberfläche - Player}
\label{fig:MockupPlayerVersion}
\end{figure}

%\begin{figure}[h!]
%\includegraphics[width=.8\textwidth,center]{MockupPlayerVersion2.pdf}
%\caption{\label{fig:MockupPlayerVersion22}Zweiter Entwurf des Players}
%\end{figure}


%%%%%%%%%%
\subsubsection{Galerie}
Die Galerie soll dazu dienen, einen Überblick über die vorhanden Zusatzinhalte zu bieten. Die Zusatzinhalte stellen alle einen grafischen Inhalt dar. Dementsprechend kann jeder Zusatzinhalt durch ein kleines Vorschaubild repräsentiert werden. 
Eine weitere Grundfunktionalität einer Galerie ist die vergrößerte Anzeige der in der Vorschau dargestellten Inhalte, die auch in unserer Galerie zur Verfügung stehen soll. Beim Erstellen des Designs muss zusätzlich auch die Anforderung der Rückkopplung zum Player bedacht werden (siehe Abschnitt \ref{sub:AnforderungenOberflaeche}). 
%Neben der Grundfunktionalität einer Galerie, dass der ausgewählte Zusatzinhalt vergrößert angezeigt werden können soll, muss beim Erstellen des Designs auch die Anforderung der Rückkopplung zum Player bedacht werden (siehe Abschnitt \ref{sub:AnforderungenOberflaeche}). 

Die einfachste Umsetzung der Galerie ist eine Darstellung der Zusatzinhalte in einem einfachen Grid  mit Scrollbalken, wie es in Abbildung \ref{fig:MockupGalerieGrid} zu sehen ist. Zusätzlich wird das Grid um zwei Buttons für eine vergrößerte Ansicht des Zusatzinhalts sowie für die Rückkopplung zum Player ergänzt. Um eine dieser beiden Aktionen auszuführen, müsste also der gewünschte Zusatzinhalt markiert und der entsprechende Button betätigt werden.

%\begin{figure}[h!]
%\includegraphics[width=.5\textwidth,center]{MockupGalerieGrid.pdf}
%\caption{\label{fig:MockupGalerieGrid}Garlerie als einfaches Grid}
%\end{figure}

Diese Variante hat den Vorteil, dass besonders viele Zusatzinhalte gleichzeitig angezeigt werden können. Auf der anderen Seite erhalten wir aber keinerlei Informationen zu den Zusatzinhalten. In der zweiten Variante, bei dem das Grid um einen Bereich für Details ergänzt wurde, kann man zumindest die Details des ausgewählten Zusatzinhaltes einsehen. Diese in Abbildung \ref{fig:MockupGalerieGridErweitert} erkennbaren Details sind natürlich von den vorhandenen Metadaten abhängig. Nachteil ist in diesem Fall aber, dass, durch den Bereich für die Details, bei gleicher Größe der Galerie weniger Zusatzinhalte zur selben Zeit dargestellt werden können. Das führt dazu, dass die Verwendung des Scrollbalkens häufiger notwendig wird.

%\begin{figure}[h!]
%\includegraphics[width=.5\textwidth,center]{MockupGalerieGridErweitert.pdf}
%\caption{\label{fig:MockupGalerieGridErweitert}Galerie als Grid mit Bereich für Details}
%\end{figure}

Bei einer Darstellung der Zusatzinhalte als Kacheln, wie in Abbildung \ref{fig:MockupGalerieKacheln} zu sehen, können gleichzeitig für alle vorhandenen Zusatzinhalte die Details angezeigt werden. Durch diese Art der Darstellung passen jedoch noch weniger Zusatzinhalte auf die gleiche Fläche.

%\begin{figure}[h!]
%\includegraphics[width=.5\textwidth,center]{MockupGalerieKacheln.pdf}
%\caption{\label{fig:MockupGalerieKacheln}Galerie mit Darstellung in Kachelform}
%\end{figure}

Eine besonders schicke Art der Darstellung wäre die des Cover Flows, bekannt aus verschiedenen Musikplayern. In Abbildung \ref{fig:MockupGalerieCoverFlow} ist zu erkennen, dass auch hier ausschließlich Details des aktuell ausgewählten Zusatzinhaltes sichtbar sind. Des Weiteren hat diese Darstellungsweise den großen Nachteil, dass auch nicht auf einen Blick alle verfügbaren Zusatzinhalte ersichtlich sind. Dies erschwert das Durchsuchen der Zusatzinhalte ungemein. Somit ist diese Art der Darstellung zwar schön anzusehen, aber nicht sonderlich gebrauchstauglich im Zusammenhang dieser Arbeit.

%\begin{figure}[h!]
%\includegraphics[width=.5\textwidth,center]{MockupGalerieCoverFlow.pdf}
%\caption{\label{fig:MockupGalerieCoverFlow}Galerie als Cover Flow}
%\end{figure}

Letztlich stellt sich die optimierte Variante der Kachel Darstellung aus Abbildung \ref{fig:MockupGalerieFinal} als beste Lösung heraus. Die Optimierung besteht daraus, dass die beiden Buttons obsolet gemacht werden. Dies kann zum einen erreicht werden, indem die vergrößerte Darstellung durch einen Klick auf die Abbildung des Zusatzinhaltes ausgelöst wird. Zum anderen bietet der Bereich der Details noch ausreichend Platz, um hier die Funktion zur Rückkopplung an den Player einzufügen. Durch diese Verbesserungen wird nicht nur mehr Platz geschaffen, sondern auch die Benutzerfreundlichkeit erhöht, indem der Vorgang zum Anzeigen der vergrößerten Ansicht beziehungsweise des Springens an den entsprechenden Zeitpunkt jeweils um einen Klick reduziert wurde.

\begin{figure}[h!]
\begin{subfigure}[c]{0.5\textwidth}
\includegraphics[width=0.8\textwidth,center]{MockupGalerieGrid.pdf}
\subcaption{Galerie als einfaches Grid}
\label{fig:MockupGalerieGrid}
\end{subfigure}%
\begin{subfigure}[c]{0.5\textwidth}
\includegraphics[width=0.8\textwidth,center]{MockupGalerieGridErweitert.pdf}
\subcaption{Galerie als Grid mit Bereich für Details}
\label{fig:MockupGalerieGridErweitert}
\end{subfigure}
\par\bigskip
\begin{subfigure}[c]{0.5\textwidth}
\includegraphics[width=0.8\textwidth,center]{MockupGalerieKacheln.pdf}
\subcaption{Galerie mit Darstellung in Kachelform}
\label{fig:MockupGalerieKacheln}
\end{subfigure}%
\begin{subfigure}[c]{0.5\textwidth}
\includegraphics[width=0.8\textwidth,center]{MockupGalerieCoverFlow.pdf}
\subcaption{Galerie als Cover Flow}
\label{fig:MockupGalerieCoverFlow}
\end{subfigure}
\par\bigskip
\begin{subfigure}[c]{\textwidth}
\includegraphics[width=0.4\textwidth,center]{MockupGalerieFinal.pdf}
\subcaption{Finale Version der Galerie}
\label{fig:MockupGalerieFinal}
\end{subfigure}
\caption{Benutzeroberfläche - Galerie}
\label{fig:MockupGalerie}
\end{figure}


%\begin{figure}[h!]
%\includegraphics[width=.5\textwidth,center]{MockupGalerieFinal.pdf}
%\caption{\label{fig:MockupGalerieFinal}Finale Version der Galerie}
%\end{figure}

%%%%%%%%%%
\subsubsection{Kommentarsektion}
Die Kommentarsektion ist für die Anzeige der öffentlichen Kommentare sowie der persönlichen Notizen zuständig. Zusätzlich muss eine Suchmaske auf Basis der Anforderungen aus Abschnitt \ref{sub:AnforderungenKommentarsektion} in die Oberfläche integriert werden.

Abbildung \ref{fig:MockupKommentarsektionVersion1} zeigt eine erste Version der Kommentarsektion. Neben der Suchmaske im Kopfbereich befinden sich zwei Checkboxen. Diese ermöglichen es dem Betrachter nach öffentlichen Kommentaren und persönlichen Notizen zu filtern. Im benachbarten Dropdown-Menü kann die Grundlage der Sortierung bestimmt werden. Die Sortierung kann nach Erstellungsdatum beziehungsweise nach Zeitpunkt der Annotation innerhalb des Hyperaudio-Dokuments erfolgen. Unterhalb dieser Funktionen befindet sich die Anzeige der Kommentare und Notizen. Sowohl bei Kommentaren als auch bei Notizen wird neben dem Erstellungsdatum auch der Annotationszeitpunkt festgehalten. Dieser wird als Link umgesetzt, sodass bei einem Klick die Rückkopplung an den Player erfolgen kann. Bei Kommentaren gibt es nach Betätigung der \textit{Antworten}-Schaltfläche noch eine zusätzliche Eingabemaske zum Verfassen von Antworten. Persönliche Notizen werden durch ein Schloss-Symbol hinter dem Erstellungsdatum visualisiert. Zusätzlich befinden sich noch jeweils zwei Buttons zum Bearbeiten und Löschen auf der rechten Seite einer Notiz.

%\begin{figure}[h!]
%\includegraphics[width=\textwidth,center]{MockupKommentarsektionVersion1.pdf}
%\caption{\label{fig:MockupKommentarsektionVersion1}Erste Version der Kommentarsektion}
%\end{figure}

Im nochmals verbesserten Design der Kommentarsektion, welches in Abbildung \ref{fig:MockupKommentarsektionFinal} abgebildet ist, werden die Antworten auf Kommentare eingerückt dargestellt. Diese Darstellung führt zu einer besseren Übersichtlichkeit und ist auch aus anderen modernen Anwendungen bekannt.

\begin{figure}[h!]
\begin{subfigure}[c]{\textwidth}
\includegraphics[width=\textwidth,center]{MockupKommentarsektionVersion1.pdf}
\subcaption{Erste Version}
\label{fig:MockupKommentarsektionVersion1}
\end{subfigure}
\par\bigskip
\begin{subfigure}[c]{\textwidth}
\includegraphics[width=\textwidth,center]{MockupKommentarsektionFinal.pdf}
\subcaption{Finale Version}
\label{fig:MockupKommentarsektionFinal}
\end{subfigure}
\caption{Benutzeroberfläche - Kommentarsektion}
\label{fig:MockupKommentarsektion}
\end{figure}

%\begin{figure}[h!]
%\includegraphics[width=\textwidth,center]{MockupKommentarsektionFinal.pdf}
%\caption{\label{fig:MockupKommentarsektionFinal}Finale Version der Kommentarsektion}
%\end{figure}


%%%%%%%%%%
\subsubsection{Zusammenführen der Elemente}
Im ersten Schritt führen wir die jeweils favorisierten Elemente in ein Layout zusammen. Dabei orientieren wir uns zunächst an unserer grobe Skizze aus Kapitel \ref{cha:analyse}. Wie nun in Abbildung \ref{fig:MockupSeiteLayoutVersion1} zu erkennen ist, ist die Kommentarsektion so in die Breite gezogen, dass das Lesen der Inhalte unangenehm werden kann. Aus diesem Grund wird in der finalen Version (siehe Abbildung \ref{fig:MockupSeiteLayoutFinal}) die Breite der Kommentarsektion auf die Breite des Players beschränkt. Dies hat zeitgleich zur Folge, dass der nun vorhandene freie Platz für die Galerie verwendet werden kann. Spätestens hiermit wird der Nachteil der gewählten Darstellungsweise der Galerie egalisiert, da nun ausreichend viele Zusatzinhalte ohne die Verwendung des Scrollbalkens eingesehen werden können.

%\begin{figure}[h!]
%\includegraphics[width=\textwidth,center]{MockupSeiteLayoutVersion1.pdf}
%\caption{\label{fig:MockupSeiteLayoutVersion1}Erstes Layout der Seite für Hyperaudio-Dokumente}
%\end{figure}

%\begin{figure}[h!]
%\includegraphics[width=\textwidth,center]{MockupSeiteLayoutFinal.pdf}
%\caption{\label{fig:MockupSeiteLayoutFinal}Finales Layout der Seite für Hyperaudio-Dokumente}
%\end{figure}

\begin{figure}[h!]
\begin{subfigure}[c]{\textwidth}
\includegraphics[width=0.9\textwidth,center]{MockupSeiteLayoutVersion1.pdf}
\subcaption{Erste Version}
\label{fig:MockupSeiteLayoutVersion1}
\end{subfigure}
\par\bigskip
\begin{subfigure}[c]{\textwidth}
\includegraphics[width=0.9\textwidth,center]{MockupSeiteLayoutFinal.pdf}
\subcaption{Finale Version}
\label{fig:MockupSeiteLayoutFinal}
\end{subfigure}
\caption{Benutzeroberfläche - Layout der Seite für Hyperaudio-Dokumente}
\label{fig:MockupSeiteLayout}
\end{figure}

%%%%%%%%%%
%\subsection{Administrationsseite eines Hyperaudio-Dokuments}
%\dots


%%%%%%%%%%
%\subsection{Moodle-Oberfläche im Allgemeinen}
%\dots


%%%%%%%%%%
\section{Datenbankentwurf}
\dots


%%%%%%%%%%
\section{Definition des Schnittstellenformats für Hyperaudio-Dokumente}
\dots

%%%%%%%%%%
\section{Zusammenfassung}
\dots
